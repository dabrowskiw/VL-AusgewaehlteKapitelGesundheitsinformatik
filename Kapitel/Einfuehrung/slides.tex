\setbeamertemplate{background}[bgfirst]
\setbeamertemplate{footline}[first]
\subtitle{Bioinformatik}
\titlegraphic{Kapitel/Einfuehrung/Bilder/saelogo.jpg}
\begin{frame}[noframenumbering]
    \titlepage
    \begin{textblock}{10}(4.75,15)
    \cite{saelogo}
    \end{textblock}
\end{frame}
\setbeamertemplate{footline}[presentationbody] 
\setbeamertemplate{background}[bgbody]

\begin{frame}{Vorstellung}
	\begin{itemize}
		\item<2-> Kontakt: Piotr.Dabrowski@htw-berlin.de, Sprechstunde Montag 10-11 - gerne nutzen!
		\item<2-> Github: https://github.com/dabrowskiw/
		\item<3-> Geboren 1981 in Warschau
		\item<3-> Studium der Biotechnologie \& Informatik an der TU Berlin
		\item<3-> Promotion über Auswertung von Hochdurchsatzdaten für Virus-Diagnostik
		\item<3-> Aufbau der bioinformatischen Analytik für das NGS-Labor des RKI
		\item<3-> Aufbau der Bioinformatics Core Facility am RKI
		\item<4-> Seit WS 2019/2020 an der HTW
		\item<5-> Hang zum Experimentieren in der Vorlesung - Feedback erwünscht!
		\item<6-> Der Todesstern \& Mini-Whiteboards
	\end{itemize}
\end{frame}

\begin{frame}{Benotung}
	\begin{itemize}
		\item Projektaufgabe (Abgabe 29.01.2020 24:00)
		\item Klausur
		\only<2->{
			\item Verbesserungsvorschläge
			\begin{itemize}
				\item 2.5\% pro umgesetzten Vorschlag, maximal 5\% pro Jahr
				\item Als pull request: Doppelte Punktzahl
			\end{itemize}
		}
	\end{itemize}
\end{frame}
