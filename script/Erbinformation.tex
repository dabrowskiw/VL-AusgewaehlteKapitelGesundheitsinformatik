\section{Erbinformation}
  Jeder lebende Organismus (Tiere, Bakterien, Pilze, Amöben, Pflanzen) sowie Viren\footnote{Da Viren sich nicht eigenständig vermehren können sondern für viele kritische Vorgänge auf die während der Infektion gekaperte Zellmaschinerie des Wirts angewiesen sind, besteht kein Konsens darüber, ob Viren leben oder nicht} besitzt Erbinformation. Diese beschreibt, welche biochemischen Prozesse der Organismus in der Lage ist, durchzuführen. Daraus ergibt sich das Aussehen und Verhalten des jeweiligen Organismus. Alles was in einem Organismus geschieht - von der Verstoffwechslung von Zucker zur Energiegewinnung über den Aufbau von Zellmembranen und extrazellulären Matrices (wie z.B. Knochen) bis hin zu der unterschiedlichen Ausdifferenzierung von Zellen in unterschiedlichen Bereichen des Körpers aufgrund der Einwirkung von Botenstoffen - ist auf solche biochemischen Prozesse zurückzuführen. 

\subsection{DNA}
  Der Träger der Erbinformation sind Ketten von Nukleinbasen. Die Verkettung erfolgt über ein Rückgrad von Zuckern und Phosphosäureestern, wodurch die gesamte Kette eine Säure ist (Nukleinsäure, Englisch nucleic acid). In manchen Viren handelt es sich bei den Zuckermolekülen um Ribose, die gesamte Kette wird dann als Ribonukleinsäure (RNS, auch im Deutschen Sprachraum ist aber die englische Abkürzung \textbf{RNA} für ribonucleic acid gängiger) bezeichnet. In anderen Viren sowie allen sonstigen Organismen kommen modifizierte Zuckermoleküle (Desoxyribose\footnote{Es fehlt im Vergleich zur Ribose ein Sauerstoffmolekül, daher der Name: Des für ohne und Oxy für Sauerstoff}) zum Einsatz, die Kette heißt dann entsprechend Desoxyribonukleinsäure (DNS, bzw. Englisch \textbf{DNA}). 

  Die Reihenfolge der \textbf{Nukleinbasen} definiert - wie die Abfolge von Zeichen in einer Textdatei - den Inhalt der Erbinformation. Dabei ist das Alphabet der DNA deutlich kleiner als der ASCII-Zeichensatz. Es gibt nur fünf unterschiedliche Nukleinbasen, die der Einfachheit halber bei der textuellen Repräsentation einer DNA-Sequenz mit den Anfangsbuchstaben ihrer formalen Namen dargestellt werden: \textbf{Adenin (A), Cytosin (C), Guanin (G), Thymin (T) und Uracil (U)}. Dabei kommen in der DNA ausschließlich Adenin, Cytosin, Guanin und Thymin vor, in der RNA wird statt Thymin Uracil eingesetzt. Entsprechend werden A, C, G und T als \textbf{DNA-Basen} und A, C, G und U als \textbf{RNA-Basen} bezeichnet. 

  Nukleinbasen haben die Eigenschaft, dass sie passende Paare bilden, die aneinander binden können - Adenin und Thymin (bzw. im Fall von RNA Uracil) sowie Cytosin und Guanin passen jeweils zueinander. Entsprechend werden diese zwei Paare jeweils als \textbf{komplementäre Basen} bezeichnet. Diese Eigenschaft wird bei der Zellteilung verwendet, um die in einer Zelle vorliegende DNA zu kopieren. Die \textbf{Primase}, ein spezialisiertes Protein, kann ein kleines Stück DNA-Einzelstrang kopieren\footnote{Die Kopie ist zunächst RNA und wird nachher von einer anderen Polymerase abgebaut und durch DNA ersetzt, dies ist aber im Kontext dieser Vorlesung ein Randdetail.}, indem sie an ein paar aufeinanderfolgende Nukleinbasen die jeweils komplementären Basen legt und diese mit einem Rückgrad verbindet. Diese kurze Kopie heißt \textbf{Primer}\footnote{Vom Lateinischen primarum - fundamental, essentiell, an erster Stelle stehend.}, denn sie wird von der \textbf{Polymerase} - einem weiteren spezialisierten Protein - erkannt und als Startpunkt für die DNA-Replikation verwendet. Die Polymerase setzt an der Stelle, an der ein Doppelstrang zu einem Einzelstrang wird, an (also am Ende eines Primers), und kopiert den Einzelstrang Base für Base indem sie jeweils die komplementären Basen anbaut und somit den Doppelstrang erweitert. Die Richtung, in der dieser Kopiervorgang geschieht, ist durch die chemische Struktur des Rückgrads definiert. Jedes Element des Rückgrads enthält ein Zuckermolekül, dessen Kohlenstoffatome der Reihe nach durchnummeriert werden. Die vorhergehende Nukleinbase ist über das dritte Kohlenstoffatom angebunden, die nächste über das fünfte. Demenstprechend spricht man vom \textbf{3'-Ende} und vom \textbf{5'-Ende} eines DNA-Moleküls. Die Polymerase kann DNA nur in \textbf{3'$\rightarrow$5'-Richtung ablesen} und in \textbf{5'$\rightarrow$3'-Richtung synthetisieren}. Entsprechend entsteht bei der Replikation ein DNA-Doppelstrang, bei dem die Einzelstränge jeweils zueinander gegenläufig aus komplementären Basen aufgebaut sind und als zueinander \textbf{revers-komplementär} bezeichnet werden.

  In den meisten Organismen, in denen DNA der Träger der Erbinformation ist, liegt die DNA nicht einfach als einzelner Strang vor\footnote{Es gibt einige Viren, die einzelsträngige DNA verwenden}. Stattdessen sind zwei gegenläufige Einzelstränge - der eine direkt an dem anderen synthetisiert, wie oben beschrieben - miteinander zu einem \textbf{DNA-Doppelstrang} verbunden. Durch Spannungen im Rückgrad\footnote{Die Basen sind durch den geometrischen Aufbau der Rückgrad-Moleküle nicht exakt parallel zueinander sondern jeweils um jeweils ca. 30 Grad gegeneinander verdreht} sind dabei die Einzelstränge umeinander verdrillt, der Doppelstrang liegt in Form der berühmten \textbf{DNA-Doppelhelix}\footnote{Entsprechend muss vor der oben beschriebenen Replikation dieser Strang entwunden werden. Dies geschieht durch ein weiteres Protein names Helikase.} vor. 

\subsection{Transkription und Translation}
Die DNA an sich ist reiner Träger der Erbinformation. Die tatsächliche Arbeit wird von \textbf{Proteinen} durchgeführt. Proteine können z.B. Bausteine in Zellstrukturen sein oder biochemische Reaktionen katalysieren, im zweiten Fall werden sie als \textbf{Enzyme} bezeichnet. Diese setzen sich, ähnlich wie die DNA, aus Ketten von Bausteinen - bei Proteinen sind es die \textbf{Aminosäuren} - zusammen. Die Reihenfolge der Aminosäuren in einem Protein, und damit seine Funktion, werden von der DNA bestimmt. Die DNA enthält also die Baupläne für alle Proteine, die ein Organismus bilden kann. Der tatsächliche Aufbau von Proteinen auf Basis der DNA-Sequenz findet in zwei Schritten statt: Der \textbf{Transkription} (Kopieren des Teils der DNA, auf dessen Basis ein Protein erstellt werden soll in messenger-RNA) und \textbf{Translation} (Erstellen eines Proteins aus der messenger-RNA).

Proteine können nicht direkt auf Basis von DNA erstellt werden. Zunächst muss die relevante Sequenz in \textbf{messanger-RNA} (\textbf{mRNA}) umgeschrieben werden. Das erlaubt eine parallelisierung der Proteinsynthese (aus einem Stück DNA können hintereinander mehrere Stücke mRNA generiert werden, auf deren Basis gleichzeitig mehrere Kopien des Proteins synthetisiert werden), wodurch ein Organismus schneller durch die Herstellung einer großen Menge eines Proteins auf Veränderungen reagieren kann, als es bei direkter Synthese aus der DNA möglich wäre. Außerdem bietet der Zwischenschritt über die mRNA zusätzliche Regulierungsmöglichkeiten (z.B. Modifikationen der mRNA, Abbau oder Stabilisierung von mRNA). Die Transkription wird ähnlich wie die DNA-Replikation durchgeführt. Ein spezialisiertes Protein, die \textbf{RNA-Polymerase}\footnote{Genauer: DNA-abhängige RNA-Polymerase, als Unterscheidung zur RNA-abhängigen RNA-Polymerase, die RNA zu RNA kopiert}, baut zu einem Teil eines DNA-Einzeltranges Base für Base einen revers-komplementären RNA-Einzelstrang auf. Im Gegensatz zur DNA-Polymerase benötigt die RNA-Polymerase allerdings keine Helikase - sie ist in der Lage, den DNA-Doppelstrang selber zu entwinden. Zudem ist sie auf keinen Primer angewiesen sondern kann direkt vom DNA-Einzelstrang aus mit dem Kopieren beginnen. Stattdessen wird der Startpunkt für die Transkription von anderen Proteinen vorgegeben, die erkennen, welche Proteine unter den jeweiligen Umständen in der Zelle benötigt werden und die RNA-Polymerase zu den entsprechenden Bereichen in der DNA führen. 

Nachdem ein Bereich der DNA in mRNA umgeschrieben wurde, kann auf Basis seiner Sequenz ein Protein synthetisiert werden. Die Synthese wird durch \textbf{Ribosome}, Komplexe aus RNA und Proteinen, durchgeführt. Ein Ribosom setzt am Anfang der mRNA an und verknüpft Aminosäuren in einer von der Reihenfolge der Basen auf der mRNA vorgegebenen Reihenfolge miteinander. Da es allerdings 20 unterschiedliche Aminosäuren gibt, die zu einem Protein verkettet werden können, und nur 4 RNA-Basen, reicht eine einzelne Base nicht um eine Aminosäure zu identifizieren. Stattdessen definieren immer drei aufeinanderfolgende Basen (als \textbf{Triplet} bzw. \textbf{Codon} bezeichnet) eine Aminosäure. Zudem gibt es ein Codon (\textbf{Start-Codon}: AUG), welches neben einer Aminosäure (Methionin) auch den Start einer Proteinsequenz markiert, und drei Codons (\textbf{Stop-Codons}: UAA, UAG und UGA), die das Ende einer Proteinsequenz markieren. Das Erreichen eines Stop-Codons durch das Ribosom führt zum Beenden der Aminosäure-Synthese. 

Im Kontext der Translation wird von \textbf{Open Reading Frames} (\textbf{ORF}) gesprochen: Ein ORF ist eine Abfolge von Basen, die mit dem Start-Codon beginnt und eine gannzahlige Anzahl von Triplets später mit einem Stop-Codon endet. Nicht jeder ORF codiert tatsächlich ein vom Organismus hergestelltes Protein - es können in Basenfolgen mit anderer biologischer Bedeutung auch Start- und Stop-Codons vorkommen - aber ein Protein kann nur von einem ORF codiert werden\footnote{In Viren und Bakterien stimmt das so direkt. In komplexeren Organismen wird zum Teil die mRNA vor der Translation modifiziert indem Teile, deren Länge nicht zwingend ein Vielfaches von 3 ist, ausgeschnitten werden (Stichworte: Introns und Exons, Splicing), so dass das Erkennen von ORFs in der unmodifizierten genomischen Sequenz nicht trivial ist.}. Ein ORF, der tatsächlich ein Protein codiert, wird als \textbf{Coding Sequence} (\textbf{CDS}) bezeichnet. 

\subsection{Mutationen}

Der Hauptgrund, weshalb Organismen ihre DNA mittels Polymerase kopieren, ist die Vermehrung. Teilt sich beispielsweise ein Bakterium, müssen beide Tochterzellen jeweils eine Kopie der DNA erhalten. Der Kopiervorgang ist allerdings nicht fehlerfrei: Manchmal baut die Polymerase während der Synthese des neuen Stranges eine falsche (nicht zur entsprechenden Base im Ursprungsstrang komplementäre) Base ein\footnote{Das eigentliche Falsch-Einbauen geschieht vergleichsweise häufig, DNA-Polymerasen haben allerdings eine sogenannte Proofreading-Funktionalität, die die letzte eingebaute Base nochmals kontrolliert und sie, falls sie falsch ist, wieder abspaltet und einen neuen Versuch startet. Dadurch wird die Fehlerrate extrem reduziert. RNA-abhängige RNA-Polymerasen besitzen diese Funktionalität nicht, daher haben RNA-Viren (Viren, die ihre Erbinformation in RNA speichern, beispielsweise HIV) vergleichsweise hohe Mutationsraten, wodurch sie sich besonders schnell verändern und u.A. entsprechend schnell Resistenzen gegen Medikamente entwickeln können}. Ein solcher Fehler führt dazu, dass die Erbinformation einer Tochterzelle sich von der Erbinformation der Elternzelle unterscheidet und wird als \textbf{Mutation} bezeichnet. Solche Mutationen treten \textbf{rein zufällig} auf, sie betreffen nicht gezielt bestimmte Regionen des Genoms. Dabei kann es, wenn eine Mutationen in einer CDS stattfindet, je nach dem welche Base in einem Codon betroffen ist, zu einer Auswirkung auf das entsprechende Protein kommen (z.B. ACG=Threonin zu AAG=Lysin) oder aufgrund der Redundanz des genetischen Codes auch nicht (z.B. ACG=Threonin zu ACT=Threonin, bezeichnet als \textbf{stumme Mutation}). Ist die Mutation nicht stumm, verändert sich durch die veränderte Aminosäuren-Abfolge die Funktionalität des Proteins (meist geringfügig, manchmal aber auch massiv). Die Folge kann für den Organismus positiv (z.B. Entstehen von Resistenz gegen ein aktuell in der Umwelt vorhandenes Antibiotikum, beispielsweise indem ein Protein des Bakteriums in die Lage versetzt wird, dieses Antibiotikum abzubauen) oder negativ (z.B. Verlust der Fähigkeit, Zucker unter Gewinnung von Energie zu verbrennen, durch Funktionsverlust eines Proteins im Stoffwechselkreislauf) sein. Organismen, die unter den aktuellen Bedingungen vorteilhafte Mutationen tragen, vermehren sich schneller, als solche die nachteilhafte Mutationen tragen, und verdrängen letztere innerhalb weniger Generationen entsprechend.

Es gibt genomische Regionen, die für einen Organismus in jeder Situation sehr wichtige und hochoptimierte Proteine codieren. Es ist extrem unwahrscheinlich, dass Bedingungen eintreten, unter denen eine Mutation in einer dieser Regionen vorteilhaft für den Organismus ist, entsprechend werden Organismen die dort Mutationen tragen schnell wieder aus der Population verdrängt. Solche Regionen verändern sich also kaum und werden als \textbf{konservierte Genomregionen} bezeichnet. Andere Regionen tendieren stärker dazu, unter veränderlichen Umweltbedingungen (Veränderungen der Temperatur, Nährstoffverfügbarkeit, aber auch Übergang zu einem anderen Wirtskörper bei Pathogenen) Mutationen anzusammeln. Anhand der Anzahl von Mutationen, in denen sich Organismen voneinander unterscheiden, kann man auf ihren Verwandheitsgrad (z.B. direkte Ansteckung zwischen Patienten vs. weit voneinander entfernte Patienten in einer langen Infektionskette) schließen.

\subsection{Empfohlene Youtube-Videos}
\begin{description}[align=left]
	\item [DNA-Struktur] \href{https://www.youtube.com/watch?v=o\_-6JXLYS-k}{https://www.youtube.com/watch?v=o\_-6JXLYS-k}
	\item [DNA-Replikation] \href{https://www.youtube.com/watch?v=0Ha9nppnwOc}{https://www.youtube.com/watch?v=0Ha9nppnwOc} (Lagging strand und Okazaki-Fragmente nur zur Information, nicht prüfungsrelevant)
	\item [Transkription und Translation] \href{https://www.youtube.com/watch?v=2BwWavExcFI}{https://www.youtube.com/watch?v=2BwWavExcFI} (Splicing, Endoplasmatisches Retikulum und Golgi-Apparat nur zur Information, gibt es in Bakterien nicht)
\end{description}
  
\subsection{Kontrollfragen}
\begin{enumerate}
	\item Wie viele Nucleinbasen gibt es? Wie heißen sie?
	\item Was sind komplementäre Basen?
	\item Welches Enzym kann ausgehend von einem gebundenen Primer einen DNA-Strang kopieren?
	\item Welche Bedingungen müssen erfüllt sein, damit die Polymerase ein Stück DNA kopieren kann?
	\item Kann die in einem Bakterium vorliegende DNA in ihrer natürlichen Form direkt von einer Polymerase kopiert werden?
	\item Ist der Gehalt an Erbinformation zwischen einem DNA-Einzelstrang und einem DNA-Doppelstrang unterschiedlich?
	\item Wofür werden Start- und Stop-Codons benötigt?
	\item Was ist ein Open Reading Frame?
	\item Lässt sich eine Aminosäure-Sequenz eindeutig wieder in eine DNA-Sequenz zurückübersetzen?
\end{enumerate}

