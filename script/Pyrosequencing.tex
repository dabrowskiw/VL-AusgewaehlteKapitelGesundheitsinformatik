\subsubsection{Pyrosequenzierung}

Eine modernere Sequenziermethode ist die \textbf{Pyrosequenzierung}. Ähnlich wie bei der Sanger-Sequenzierung ist das Ausgangsmaterial ein PCR-Produkt, und die Sequenzierreaktion selber ähnelt der PCR-Reaktion. Im Gegensatz zur Sanger-Sequenzierung wird allerdings hier die Geschwindigkeit der Synthese des Gegenstranges streng kontrolliert: Zunächst werden zu dem zu sequenzierenden PCR-Produkt nur Primer, Polymerase sowie ein paar weitere, für die spätere Signalgenerierung notwendige Enzyme hinzugegeben. Dann wird nur ein Typ Nukleitid hinzugefügt - also nur A, T, G oder C, anstatt wie bei der PCR oder Sanger-Sequenzierung ein Mix aller Nukleotide. Kann dieses Nukleotid von der Polymerase eingebaut werden, werden die bei dem Einbau entstehenden Nebenprodukte von den zusätzlichen Enzymen verwendet, um ein Lichtsignal zu generieren\footnote{Die hier verwendeten Enzyme sind die gleichen, die von Glühwürmchen verwendet werden, um Licht zu generieren. Man macht sich zu Nutze, dass das bei der Verknüpfung zweier Basen anfallende Nebenprodukt das gleiche ist, welches die Glühwürmchen auch als Ausgangspunkt in ihrer Lichterzeugungs-Reaktion verwenden. Dieser Ausgangsstoff ist das Pyrophosphat, daher der Name der Sequenziermethode.}. Dieses Lichtsignal kann mit einer Kamera aufgenommen werden. Danach werden die überschüssigen Nukleotide entfernt und ein anderes Nukleotid wird hinzugefügt. Daraus, bei welcher Nukleotidzugabe Lichtsignale entstehen, kann die Sequenz rekonstruiert werden. Die Pyrosequenzierung erlaubt es zwar, im Gegensatz zur Sanger-Sequenzierung, den Sequenziervorgang in echtzeit mitzuverfolgen, allerdings können mit dieser Methode nur ca. 200 Basen lange Sequenzen generiert werden, bevor das Signal zu Rauschen degeneriert. 

