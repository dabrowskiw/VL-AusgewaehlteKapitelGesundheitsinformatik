\section{De novo-Assemblierung}

\subsection{Overlap-Layout-Consensus}

\subsection{De Bruijn-Graph}

\subsection{Empfohlene Youtube-Videos}
\begin{description}[align=left]
	\item [Overlap-Graph] \href{https://www.youtube.com/watch?v=yPJ7yHRk2OI}{https://www.youtube.com/watch?v=yPJ7yHRk2OI}
	\item [De-Bruijn-Graph] \href{https://www.youtube.com/watch?v=TNYZZKrjCSk}{https://www.youtube.com/watch?v=TNYZZKrjCSk} (Eulerkreisproblem nur als theoretischer Hintergrund, nicht klausurrelevant)
	\item [Contigs] \href{https://www.youtube.com/watch?v=0Ho2\_\_cFsVY}{https://www.youtube.com/watch?v=0Ho2\_\_cFsVY}
	\item [Gesamtüberblick de novo-Assemblierung] \href{https://www.youtube.com/watch?v=V2xIGdEkA5U}{https://www.youtube.com/watch?v=V2xIGdEkA5U} (Enthält viel Zusatzinformation, aber für das Gesamtverständnis sehr schöne Präsentation)
\end{description}

\subsection{Kontrollfragen}
\begin{enumerate}
	\item Wofür werden bei der PCR Primer benötigt?
	\item Werden bei der Sanger- oder bei der Pyrosequenzierung fluoreszent markierte Nukleotide benötigt? Welche weitere Eigenschaft (neben der fluoreszenten Markierung) haben diese modifizierten Nukleotide?
\end{enumerate}
